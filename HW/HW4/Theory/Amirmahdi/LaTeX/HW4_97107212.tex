\documentclass[12pt]{article}
\usepackage{graphicx,import}
\usepackage[svgnames]{xcolor} 
\usepackage{fancyhdr}
\usepackage{subfig}
\usepackage{hyperref}
\usepackage{enumitem}
\usepackage[many]{tcolorbox}
\usepackage{listings}
\usepackage{physics}
\usepackage{amsmath}
\usepackage{tikz}
\usepackage{mathdots}
\usepackage{yhmath}
\usepackage{cancel}
\usepackage{color}
\usepackage{siunitx}
\usepackage{array}
\usepackage{multirow}
\usepackage{amssymb}
\usepackage{gensymb}
\usepackage{tabularx}
\usepackage{extarrows}
\usepackage{booktabs}
\usetikzlibrary{fadings}
\usetikzlibrary{patterns}
\usetikzlibrary{shadows.blur}
\usetikzlibrary{shapes}

\usepackage[a4paper, total={6in, 8in} , bottom = 25mm , top = 25mm, headheight = 1.25cm , includehead,includefoot,heightrounded ]{geometry}
\usepackage{afterpage}
\usepackage{amssymb}
\usepackage{pdflscape}
\usepackage{gensymb}
\usepackage{textcomp}
\usepackage{tikz,pgfplots}
\usepackage{xecolor}
\usepackage{rotating}
\usepackage{pdfpages}
\usepackage[T1]{fontenc}
\usepackage{tikz}
\usepackage[utf8]{inputenc}
\usepackage{PTSerif} 
\usepackage{seqsplit}
\usepackage{fancyvrb}
\usepackage{mips}
\usepackage{multirow}
\usepackage{hhline}
\usepackage[edges]{forest}
\usepackage{tabularx}
\usepackage{float}
\usepackage{graphicx}
\usepackage{cprotect}
\usepackage{url}
\usepackage{listings}
\usepackage{xcolor}
\usepackage{pifont}
\newcommand{\xmark}{\ding{55}}%
\def\checkmark{\tikz\fill[scale=0.4](0,.35) -- (.25,0) -- (1,.7) -- (.25,.15) -- cycle;}

\hypersetup{
	colorlinks   = true, %Colours links instead of ugly boxes
	urlcolor     = blue, %Colour for external hyperlinks
	linkcolor    = blue, %Colour of internal links
	citecolor   = red %Colour of citations
}

\definecolor{codegreen}{rgb}{0,0.6,0}
\definecolor{codegray}{rgb}{0.5,0.5,0.5}
\definecolor{codepurple}{rgb}{0.58,0,0.82}
\definecolor{backcolour}{rgb}{0.95,0.95,0.92}
\definecolor{mGreen}{rgb}{0,0.6,0}
\definecolor{mGray}{rgb}{0.5,0.5,0.5}
\definecolor{mPurple}{rgb}{0.58,0,0.82}
\definecolor{backgroundColour}{rgb}{0.95,0.95,0.92}

\NewDocumentCommand{\codeword}{v}{
	\texttt{\textcolor{blue}{#1}}
}
\lstset{language=java,keywordstyle={\bfseries \color{blue}}}
\definecolor{dkgreen}{rgb}{0,0.6,0}
\definecolor{gray}{rgb}{0.5,0.5,0.5}
\definecolor{mauve}{rgb}{0.58,0,0.82}



\lstdefinestyle{mystyle}{
	backgroundcolor=\color{backcolour},   
	commentstyle=\color{codegreen},
	keywordstyle=\color{magenta},
	numberstyle=\tiny\color{codegray},
	stringstyle=\color{codepurple},
	basicstyle=\ttfamily\normalsize,
	breakatwhitespace=false,         
	breaklines=true,                 
	captionpos=b,                    
	keepspaces=true,                 
	numbers=left,                    
	numbersep=5pt,                  
	showspaces=false,                
	showstringspaces=false,
	showtabs=false,                  
	tabsize=2
}

\lstdefinestyle{CStyle}{
	backgroundcolor=\color{backgroundColour},   
	commentstyle=\color{mGreen},
	keywordstyle=\color{magenta},
	numberstyle=\tiny\color{mGray},
	stringstyle=\color{mPurple},
	basicstyle=\footnotesize,
	breakatwhitespace=false,         
	breaklines=true,                 
	captionpos=b,                    
	keepspaces=true,                 
	numbers=left,                    
	numbersep=5pt,                  
	showspaces=false,                
	showstringspaces=false,
	showtabs=false,                  
	tabsize=2,
	language=C
}



\lstset{ %
	language=[mips]Assembler,       % the language of the code
	basicstyle=\footnotesize,       % the size of the fonts that are used for the code
	numbers=left,                   % where to put the line-numbers
	numberstyle=\tiny\color{gray},  % the style that is used for the line-numbers
	stepnumber=1,                   % the step between two line-numbers. If it's 1, each line
	% will be numbered
	numbersep=5pt,                  % how far the line-numbers are from the code
	backgroundcolor=\color{white},  % choose the background color. You must add \usepackage{color}
	showspaces=false,               % show spaces adding particular underscores
	showstringspaces=false,         % underline spaces within strings
	showtabs=false,                 % show tabs within strings adding particular underscores
	frame=single,                   % adds a frame around the code
	rulecolor=\color{black},        % if not set, the frame-color may be changed on line-breaks within not-black text (e.g. commens (green here))
	tabsize=4,                      % sets default tabsize to 2 spaces
	breaklines=true,                % sets automatic line breaking
	breakatwhitespace=false,        % sets if automatic breaks should only happen at whitespace
	% also try caption instead of title
	keywordstyle=\color{blue},          % keyword style
	commentstyle=\color{dkgreen},       % comment style
	stringstyle=\color{mauve},         % string literal style
	escapeinside={\%*}{*)},            % if you want to add a comment within your code
	morekeywords={*,...}               % if you want to add more keywords to the set
}

\setmainfont[ExternalLocation=fonts/]{EBGaramond-Regular.ttf}




\newenvironment{changemargin}[2]{%
	\begin{list}{}{%
			\setlength{\topsep}{0pt}%
			\setlength{\leftmargin}{#1}%
			\setlength{\rightmargin}{#2}%
			\setlength{\listparindent}{\parindent}%
			\setlength{\itemindent}{\parindent}%
			\setlength{\parsep}{\parskip}%
		}%
		\item[]}{\end{list}}


\definecolor{foldercolor}{RGB}{124,166,198}

\tikzset{pics/folder/.style={code={%
			\node[inner sep=0pt, minimum size=#1](-foldericon){};
			\node[folder style, inner sep=0pt, minimum width=0.3*#1, minimum height=0.6*#1, above right, xshift=0.05*#1] at (-foldericon.west){};
			\node[folder style, inner sep=0pt, minimum size=#1] at (-foldericon.center){};}
	},
	pics/folder/.default={20pt},
	folder style/.style={draw=foldercolor!80!black,top color=foldercolor!40,bottom color=foldercolor}
}

\forestset{is file/.style={edge path'/.expanded={%
			([xshift=\forestregister{folder indent}]!u.parent anchor) |- (.child anchor)},
		inner sep=1pt},
	this folder size/.style={edge path'/.expanded={%
			([xshift=\forestregister{folder indent}]!u.parent anchor) |- (.child anchor) pic[solid]{folder=#1}}, inner xsep=0.6*#1},
	folder tree indent/.style={before computing xy={l=#1}},
	folder icons/.style={folder, this folder size=#1, folder tree indent=3*#1},
	folder icons/.default={12pt},
}

\begin{document}
	
	
%%% title pages
\begin{titlepage}
	\begin{center}
		
		\vspace*{0.7cm}
		
		\includegraphics[width=0.4\textwidth]{sharif1.png}\\
		\vspace{0.5cm}
		\textbf{ \Huge{Multicore Computing} }\\
		\vspace{0.5cm}
		\textbf{ \Large{ Assignment Four (Theory)} }
		\vspace{0.2cm}
		
		
		\large \textbf{Department of Computer Engineering}\\\vspace{0.2cm}
		\large   Sharif University of Technology\\\vspace{0.2cm}
		\large   Spring 2022 \\\vspace{0.2cm}
		\noindent\rule[1ex]{\linewidth}{1pt}
		Lecturer:\\
		\textbf{{Dr. Falahati}}
		
		
		\vspace{0.15cm}
		Name - Student Number:\\
		
		
		\textbf{{Amirmahdi Namjoo - 97107212}}
	\end{center}
\end{titlepage}
%%% title pages


%%% header of pages
\newpage
\pagestyle{fancy}
\fancyhf{}
\fancyfoot{}
\cfoot{\thepage}
\chead{ Amirmahdi Namjoo}
\rhead{\includegraphics[width=0.1\textwidth]{sharif.png}}
\lhead{Assignment Four}
%%% header of pages



\section{Question One}

\begin{enumerate}[label=\alph*)]
	\item 
	$$\text{warps} = \frac{\text{Total Thread}}{\text{Warp Size}} = \frac{1024}{32} = 32 $$

We have 1024 threads because we will have one thread per outer loop iteration (based on the question description).


\item

All 1024 threads execute instructions 1,2, and 4.

In the first iteration of \verb+j+, the value of \Verb+s+ is $1$, so threads with \Verb+i+ divisible by two will execute instruction 3. i.e., half of the threads of each warp will execute this instruction. Therefore

$$\text{SIMD Utilization} = \frac{1024 + 1024 + \frac{1024}{2} + 1024}{1024+1024+1024+1024} = \frac{7}{8}$$

\item 
All 1024 threads execute instructions 1,2 and 4.

For instruction 3, only threads with $i<512$ will run this. The difference between this one and the previous one is that in the former, half the threads of each warp were inactive. But in this case, for half of the warps, all threads are inactive, no instructions are issued, and we have no performance loss. i.e., All threads of only half of the warps execute instruction 3.

$$\text{SIMD Utilization} = \frac{1024 + 1024 + \frac{1024}{2} + 1024}{1024+1024+\frac{1024}{2}+1024} = 1$$

\item

All 1024 threads execute instructions 1,2, and 4.

For instruction 3, with $0 \leq j <5$, all 32 warps are active, and the number of active threads per warp divides by half in each iteration. The denominator, in this case, is always $4096$ with $5 \leq j < 10$, only one thread per warp is active, and some of the warps will have no active thread, and therefore scheduler will not issue any instruction for them. Therefore the denominator for these will also change.


$\text{SIMD Utilization} = \begin{cases}\frac{3072+2^{(9-j)}}{4096}, & \text { if } 0 \leq j<5 \\ \frac{3072+2^{(9-j)}}{3072+32 \times 2^{(9-j)}}, & \text { if } 5 \leq j<10\end{cases}$


\item 
All 1024 threads execute instructions 1,2 and 4.

For instruction 3, with $0 \leq j < 5$, all 32 threads in each warp are active, but not all of the warps are active. The number of warps active will halve each iteration. With $5 \leq j < 10$, only one warp (the one containing $i=0$ to $i=31$) will be active, and the number of active threads will half in each iteration. Therefore, for $0 \leq j < 5$, the value of nominator and denominator is equal, and we have $100\%$ utilization. For $5 \leq j < 10$, the nominator changes. Therefore:


$\text{SIMD Utilization} = \begin{cases} 1, & \text { if } 0 \leq j<5 \\ \frac{3072+2^{(9-j)}}{3072+32}, & \text { if } 5 \leq j<10\end{cases}$


\item 
This could not happen for $0 \leq j < 5$ because the second one has $100\%$ utilization, while the first one has utilization of less than $1$. For $5 \leq j < 10$ we can solve the equation 

$$ \frac{3072+2^{(9-j)}}{3072+32} = \frac{3072+2^{(9-j)}}{3072+32 \times 2^{(9-j)}}$$

$$\Rightarrow j = 9$$

Which is reasonable. In this case, only one thread of only one warp is active.

\item 
Code 2 will be faster; It has less intra-warp branch divergence. In other words, in most cases of Code 2, a warp is either completely active or inactive. Therefore we have high SIMD utilization, and the scheduler will not schedule completely inactive warps. In contrast, in code 1, in lots of cases, all warps are active while some threads inside warps are not active, and therefore scheduler schedules them with less than optimal utilization.

	
	
	
\end{enumerate}





\end{document}



